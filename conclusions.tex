\begin{enumerate}
% _____________________________________________________________________________________________
%   SEGUNDO DRAFT:
%    \item The complete set of proteases of the sperm whale, \textit{Physeter macrocephalus}, comprises 546 genes, including those presenting events of truncation (hence psudogenisez).
%    \item The \textsl{de novo} assembled genome of the last member of the Pinta Gal\'{a}pagos giant tortoises, \textit{Chelonoidis abingdonii}, is $2,300,749,194 bp$ long, and its authomatic anotation yielded $27,208 genes$. 
%    \item The manual annotation of said genome, reported 492 proteases (including those with pseudogenization events), thus defining Lonesome George's degradome.
%% ___________________
%% ESCOGER UNA OPCIÓN:
%%   OPCIÓN A:
%%    \item Multiple alterations to genes related to \textit{intercellular communication} in sperm whale, seem to have lead to a lesser inflammation levels, probably contributing to mitigate one of the integrative hallmarks of ageing.
%%    \item Several evolutionary events in giant Gal\'{a}pagos tortoises, both point mutations and CNVs, appear to have shifted the balance between innate and adaptative immune responses in benefit of the former, so compensating for \textit{atered intercellular communications}.
%%   OPCIÓN B:
%    \item Among the numerous genetic alterations and evolutive events provided by the study of these species, a large percentage relate to pathways and systems linked to \textit{intercellular communication}, stressing the importance of this integrative hallmark in challenging ageing (e.g. inflammageing).
%% ___________________
%    \item Distinct evolutionary features were also found associated with other hallmarks (although sometimes intertwined), namely \textit{loss of proteostasis}, \textit{stem cell exhaustion}, and \textit{derregulated nutrient sensing}.
%    \item Different strategies to counteract cancer development (as tated in Peto's paradox), were also identified in the analysis of the degradome of these animals.
%   ___________________________________________________________________________________________
%   ___________________________________________________________________________________________
%   PRIMER DRAFT:
%    \item The complete degradome of the sperm whale, \textit{Physeter macrocephalus}, contains 546 protease genes, including XX classified as pseudogenes.
%    \item We found multiple events of pseudogenization and truncation affecting proteases, including \textit{MMP12}, \textit{TPSAB1}, and \textit{MASP2}, with known roles in the immune system.
%    \item We have assembled the genome of Lonesome George, last member of \textit{Chelonoidis abingdonii}, one of the Gal\'{a}pagos giant tortoises. The automatic annotation of this assembly yielded 27,208 predicted genes.
%    \item The manual annotation of the assembled genome of Lonesome George yielded 492 protease genes, including XX classified as pseudogenes.
%    \item We have found several events of amplification, or featuring signatures of positive selection, affecting Lonesome George genes (such as \textit{VCP} or \textit{BAG2}) with known roles in \emph{loss of proteostasis}, one of the primary hallmarks of ageing.
%    \item We describe multiple specific genes presenting signatures of positive selection (such as \textit{TUBE1}, \textit{TBG1}, \textit{AHSG}, or \textit{FGF19}, linked to pathways implicated in \emph{deregulated nutrient sensing}, and \emph{stem cell exhaustion}i, secondary and integrative hallmarks respectfully.
        %    \item We report several evolutive events (such as the expansion of granzime serine proteases, the truncation of \textit{MEP1A}, or the signatures of positive selection presents in \textit{MVK}), all related to roles concerning the integrative hallmark of \emph{altered intercellular communication}.
%   ___________________________________________________________________________________________
%   
    \item The complete degradome of the sperm whale, \textit{Physeter macrocephalus}, contains 546 protease genes, including 58 classified as pseudogenes.
    \item We found multiple events of pseudogenization and truncation in sperm whales affecting proteases, including \textit{MMP12}, \textit{TPSAB1}, and \textit{MASP2}, with known roles in the immune system.
    \item We report nonsense variants in genes linked to cancer development in \textit{P. macrocephalus}, such as \textit{CASP3} and \textit{MMP7}, that are expected to eliminate their products.
    \item We have assembled the genome of Lonesome George, last member of \textit{Chelonoidis abingdonii}, one of the Gal\'{a}pagos giant tortoises. The automatic annotation of this assembly yielded 27,208 predicted genes.
%    \item The manual annotation of the assembled genome of Lonesome George yielded 515 protease genes, including 63 classified as pseudogenes.
    \item The complete degradome of Lonesome George is composed of 515 genes, 63 of which are classified as pseudogenes.
    \item Through hypothesis-driven manual annotation of the degradome of Lonesome George, we have found multiple specific variants predicted to affect the activity of proteases with known roles in \emph{altered intercellular communication}, an integrative hallmark of ageing.
    %AQUI NO HAY EJEMPLOS DE LO PRIMERO POR QUE ACABO DE COMPROBAR QUE EN NINGUN CASO SE DICEN CUALES SON LOS 8 GENES A LOS QUE SE REFIERE EL TEXTO.
        %FALTA POR RELLENAR LAS XX TAMBIÉN POR QUE QUIERO HABLAR ESE TEMA CONTIGO, VICTOR...
\end{enumerate}
