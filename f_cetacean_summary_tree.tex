\begin{sidewaysfigure}[ht!]                                                                    
\begin{center}                                                                             
\includegraphics[width=\textwidth]{figures/sperm_whale_summary.pdf}                          
    \caption[Cetacean truncations and alterations in proteases]{\footnotesize Cetacean truncations and alterations in proteases. Each mark represents an adaptative event, either truncations or duplications (represented as a doble mark), in the branch in which data suggests it occurred. Events of convergent evolution are represented by a doted line joining the pertinent marks. Those marks filled with \textcolor{myaqu1}{\textbf{green}} are those that, although mentioned in the main text, most likely played a late role once a main event of adaptation had already occured (said event will be represented by \textcolor{myora1}{\textbf{orange}} as the rest). The \textcolor{gray}{\textbf{grey}} mark represents an absence. \hsap; \bacu; \bmys; \mmon; \dleu; \ttru; \oorc; and \pmac. The tree itself is based in the relations shown by \quotes{Time tree} \cite{Kumar2017}.}
\label{f_results_sperm_whale_summary}                                                                
\end{center}                                                                               
\end{sidewaysfigure}
