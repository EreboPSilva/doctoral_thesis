Ageing, the progressive deterioration of homoeostatic capabilities of the body, is a multi-factor, quasi-universal, and poorly understood process.
In an attempt at preliminarily characterising the evolutionary strategies to counteract ageing, we have annotated, analysed and comparatively studied the degradomes of sperm whales and the iconic tortoise Lonesome George.
As a result of this, we have identified multiple genomic alterations affecting protease genes, some of which may be associated with ageing-related pathways or systems.

% HE INCLUIDO ESTE PARRAFO AQUI...
Overall, most of the predicted losses and gains of protease genes in the sperm whale mirror those described in the previously annotated genomes of minke \cite{Yim2014c} and bowhead whales \cite{Keane2015}.
Similarly, many of the findings in the Gal\'{a}pagos Giant tortoise are well conserved across Giant Tortoises and even across Archelosauria or Sauropsida.
Nevertheless, several events stand out as independent or specific, providing interesting hypotheses about the evolutionary history of these species in the context of ageing as well as specific features of adaptation to the environment.
As a \textit{driver of evolution}, the immune system has been targeted by selection in many instances in both species, each in a different way, proving once more the important role that this system plays in the history of a species. 
We have found several instances of mutations related to four of the hallmarks of ageing, namely, \textit{altered intercellular communication}, \textit{stem cell exhaustion}, \textit{derregulated nutrient sensing}, and \textit{loss of proteostasis}.
Altered intercellular communication, in particular, may play a role in explaining Peto's paradox in both species.
Finally, some results may be related to specific adaptations to the distinct history and environment of the sperm whale and the Gal\'{a}pagos giant tortoise of Pinta Island.
% ________________________________

When annotating a genome aiming to find differential features related to the biology of an organism, a key step is to compare the results with those of other species, both closely related and outgroups.
In the case of sperm whale (\textit{P. macrocephalus}), the organisms of choice were: humans as an outgroup; bowhead and Minke whales; the bottlenose dolphin; and the killer whale.
As they became available, genomic data from the narwhal and the beluga whale were included.
Overall, most of the predicted losses and gains of protease genes mirror those described in the previously annotated genomes of minke \cite{Yim2014c} and bowhead whales \cite{Keane2015}.
This comparison served as a preliminary screening for our results.
Nevertheless, several events stand out as independent or specific, providing interesting hypotheses about the evolution of sperm whales in the context of cetacean evolutionary history.
In addition to the usual selective pressure on the immune and reproductive system of mammals, the unique aquatic environment of cetaceans has prompted numerous changes affecting protease genes involved in blood homoeostasis, skin maintenance, and digestion.

The strong selective pressure on the immune system has imposed multiple variations in mammalian proteases in several species \cite{Keane2015,Puente2006,Worley2014a}.
Additionally, the underwater environment cetaceans live in poses challenges unique to this species of cetaceans, thus prompting novel immune strategies.
Also, the immune system plays an important role in cancer development.
This role may be relevant in massive mammalian organisms, given the much higher number of cells they possess compared to smaller animals.
As stated in \emph{Peto's paradox}, a similar propensity of each cell to become tumoural would lead to cancer incidences orders of magnitude higher in large mammals, which is not observed \cite{Caulin2011b}.

Therefore, conspicuous genomic changes affecting cancer-related genes in large mammals might offer interesting candidates in ageing and cancer research.
For instance, given the contribution of MMP12 to metastasis \cite{Lv2015}, the premature stop codon in cetacean orthologs may play a role in the biology of cancer in these organisms.
Considering the level of conservation among the studied cetaceans of this premature stop codon%(Figure \ref{f_results_sperm_whale_alignments_inmuno})
, this could be an early adaptive event in cetaceans evolutionary history.

In addition, a putative loss of function in matrix metalloprotease 7 (\textit{MMP7}%; Figure \ref{f_results_sperm_whale_alignments_mmp7}
), and a duplication affecting cystein protease \textit{CASP3} %(Figure \ref{f_results_sperm_whale_alignments_casp3}) 
may also merit further research.
Since \textit{MMP7}'s main role consist in degrading the extracellular matrix, which allows the cancerous cells to spread, this loss may impact the process of metastasis in sperm whales.
\textit{CASP3}, on the other hand, is known to be implicated in the proteolytic cascades associated with the apoptosis process, of great importance as an antitumoral measure.

As already mentioned, the putative loss of \textit{MMP7} would mean the total absence of any member of the matrysilin subfamily in sperm whales, the only known case in all mammals.
Of course, the \textit{MMP} family has a large number of members with partially overlapping functions, and therefore the loss of one protease can be overcome by the activity of other paralogs.
However, mice deficient in \textit{MMP7} have been shown to respond differently to challenges like re-epithelialization \cite{Swee2008}.
Interestingly, high levels of expression of this metalloprotease have also been shown to promote metastasis \cite{Li2014a,Koskensalo2011}, which suggests a putative sperm whale-specific mechanism to counteract the problems underlined in Peto’s paradox.
Indeed, a duplication in a pro-autophagic protease, such as \textit{CASP3}, could lead to an enhanced capability to enter an autophagic process, which would be a useful response mechanism against the development of tumour cells.

Adaptations related to inflammation, a process whose correct regulation is tightly linked to ageing, suggest that this response may be comparatively mild in cetaceans.
Specifically, \textit{CASP12} and \textit{TPSAB1} apparent loss of function, both point to this mitigated reaction to pathogens via down-regulating the activity of inflammatory caspases, and altering mast-cells degranulation process, respectively \cite{Abdelmotelb2014,McIlwain2015}.
Interestingly, \textit{CASP12} seems to have been lost in all cetaceans.
Despite its complex evolutionary history, the frameshift starting at {p.L101} %(figure \ref{f_results_sperm_whale_alignments_inmuno}) 
is possibly the result of a single event, early in cetacean radiation, given its high degree of conservation in this clade, suggesting a possible secondary role in the adaptation to the aquatic habitat. %Once a gene is inactivated, it is not surprising that it may accumulate deleterious mutations.
In addition, the putative loss of function of \textit{TPSAB1}, which also seems to have occurred via single, early event in the history of cetaceans, could have repercussions on the degranulation process occurring in mast cells during the first immunological response \cite{Wilcock2019}.
%Finally, the premature stop codon gained in \textit{PRSS33} could relate to a mild alteration relating to macrophages, the cell type that predominately expresses this protease.
%In this case, the apparent loss of function, while common to all cetaceans, comprises a case of convergent evolution\footnote{Meaning that the same truncation or loss is \quotes{achieved} through different mechanisms in different points of the evolutionary history of several species.} between primates and cetaceans, presenting each a different premature stop codon (or codons).
%These kind of adaptations are highly interesting because they identify some mechanism extremely fitting for an envelopment (hence adopted independently by same-environment species), and also much more malleable mechanism, able to benefit species from very diverse environments, being the latter the case here.
Finally, \textit{MASP2} is specifically truncated in sperm whales, not only by gaining a premature stop codon, but by mimicking a deleterious point mutation that causes pathologies in humans%(Figures \ref{f_results_sperm_whale_alignments_inmuno} and \ref{app_f_masp12_align})
. 
It is important to keep in mind that, according to Dobzhansky, which constitutes a pathological mutation to one species, can be a concerted, adaptive response in another \cite{Dobzhansky1958}.
This suggests that MASP2 may have been lost in a stepwise mechanism.
First, a point mutation may have altered its function in concert with other events that rendered this change innocuous.
Once the biology of this system was adapted to the loss of MASP2 activity, a second truncating event would have inactivated the gene.  
A different, independent truncation with the same expected result occurs in the bottlenose dolphin% (Figure \ref{app_f_masp12_align})
.
Intriguingly, both truncated proteins are expected to contain lectin-binding domains, but not the serine-protease domain, which suggests a possible compensating mechanism through binding of a different protease to these domains.
These events also provide a remarkable example of convergent evolution, and support the idea that loss of the serine-protease domain of MASP2 is favoured in some cetaceans.
When considered in terms of its function, these results could lead to less aggressive immune innate systems, with a diminished capacity to activate the complement path.

Taken together, these data reinforce the important role of the immune system, particularly the inflammatory response, in the evolution of cetaceans.
This is highlighted by the independent losses of important protease genes participating in this system in sperm whales and other cetaceans, specially given the putative participation of \textit{MASP2} mutations in speciation events through a complementation mechanism \cite{Kondrashov2002}.
The apparent general trend towards a milder inflammatory response may also be relevant in the study of the lower tumorigenic potential of cells in large mammals, and of course it could also yield some light into the longevity of this order.

One of the most conspicuous traits of the aquatic environment is the hydrostatic pressure and lack of net weight experienced by cetaceans.
These conditions, so different from those encountered in land, must prompt compensatory mechanisms in the control of blood pressure and coagulation to avoid haemostatic accidents.

Related to this, we reported several losses that may impact blood homoeostasis.
Specifically, two serine proteases, \textit{F12} %(figure \ref{f_results_sperm_whale_alignments_blood}) 
and \textit{KLKB1}, that relate to the Kinin-Kallikreyn System\nomenclature{KKS}{Kinnin-Kallikreyn System}, appear to have been lost in all cetaceans.
Interestingly, in the case of \textit{KLKB1}, it appears that there are several different stages of pseudogenization, including complete absences (such is the case of sperm whale), partial absences (with only two-four exons identified), and various numbers of premature stop codons in those presenting some exons.
It's worth mentioning that the automatic annotation of the Monodontidae family members yielded no results for this gene.
In all species presenting some exons, at least one stop codon is conserved.
This suggest that despite the different stages, the process that leaded to the loss of this gene could have started at the same time in an early stage of cetacean speciation.
As mentioned, both F12 and KLKB1 function as part of the KKS, a complex network of proteins and peptides involved in several biological processes, associated with exocrine glands and plasma.
In these processes, they act as potent vasodilators, increase vascular permeability, produce pain, increase lymph flow, and (in high doses) cause the accumulation of polymorphonuclear leukocytes \cite{Bader2011}.
In fact, a genome association analysis with human populations has uncovered variants of these serine proteases putatively related to increased levels of vasoactive peptides \cite{Verweij2013a}.
Therefore, their absence may be related to the extreme differences found in the aquatic versus terrestrial environment.
Moreover, the KKS also impacts the acute phase of the inflammatory process.
This suggests that the loss of \textit{F12} and \textit{KLKB1} may affect the inflammatory response along with the aforementioned variants.

Another interesting aspect of blood homoeostasis, coagulation, seems to have undergone some alterations as well.
In this sense, we reported the truncation of \textit{TMPRSS11B} and \textit{TMPRSS11F} (also known as \textit{HATL5} and \textit{HATL4} respectively) via different mechanisms%(Figures \ref{f_results_sperm_whale_alignments_blood} and \ref{app_f_tmprss11b_align})
.
In addition, factor VII (\textit{F7}), which also participates in coagulation and was reported as truncated in Mysticeti \cite{Keane2015}, appears to be functional in sperm whale.
This adds another dimension to the convoluted ways in which blood homoeostasis has evolved to adapt to a new environment, thus reflecting the intrinsic complexity of this system.
It must be noted that the sperm whale shows a natural disposition to much deeper dives than its relatives, which may constrict changes in the coagulation system.

Together, all these changes suggest that the mammalian potential for clotting and blood pressure are excessive in an aquatic environment, and these systems had to be modulated through processes that may have included the loss of proteases implicated in related proteolytic cascades.

Living underwater also sets the skin of mammals as a target of evolutionary pressure.
Several events in the degradome of cetaceans might be related to this adaptive process.
For instance, Hair-follicle cortex-related cystein protease, \textit{CAPN12} seems to have been lost through several independent events in \textit{Physeteroidea}, \textit{Delphinoidea} (including \textit{Monodontidae}, and \textit{Mysticeti}% (Figure \ref{f_results_sperm_whale_alignments_skin})
.
The annotated sequences are compatible with progressive gene losses, at different points of history.
This case of convergent evolution suggests that the loss of this gene was favoured by selection.
In addition, the loss of \textit{KLK8}% (Figure \ref{f_results_sperm_whale_alignments_skin})
, specifically linked to the maintenance of the most external layer of the skin (\textit{stratum corneum}).
Possibly relevant as well is the apparent functionality of \textit{KLK7}, lost in Mysticeti.
The loss of the catalytic  site in the case of \textit{KLK8} may have more repercussions than simply a diminished function, since the putative inactive enzyme may yet be able to sequester substrates in the cell, playing a potentially antagonistic role to other physiological cell functions.
Its independent loss in the rest of Cetacea suggests, again, that the suppression of its role may be important in the adaptation to the environment, to which the skin acts as first barrier and defence.

This complex pattern of convergent evolution suggests that skin-related proteases have played important roles in aquatic adaptation, in a process possibly influenced by the specific and somewhat contradictory requirements of heat insulation, buoyancy and diving.
Interestingly, sperm whales presents several peculiarities in the skin that may be related to the reported differences in adaptation process.
Not only does the sperm whale presents one of the relatively thinnest \textit{stratum corneum}, but it also presents an extremely wrinkled skin (sometimes referred as raisin-like) when compared to other Cetacea \cite{Sokolov1982}.
Besides this, sperm whales shed more frequently than its relatives and, as mentioned before, is one of the deepest divers among the aquatic mammals, only behind Cuvier's whale.

Regarding nutrition, different feeding strategies and diets may underlie similarly notable genetic adaptations.
Thus, multiple metalloproteases have been lost at different points during cetacean evolution, probably due to their diverse diets (fish and bigger animals, versus krill).
In short, \textit{CPA2}, \textit{CPA3}, \textit{CPO}, and \textit{CPB1}, take part in the digestive process, severing the peptidic link as part of the digestion of proteins.
As such, the specific loss of \textit{CPB1} in sperm whales and the diverse premature stop codons of \textit{CPA3} in other cetaceans could be related to the peculiar diet of this enormous creature, the only known predator of giant squids.
Unsurprisingly, when compared with the annotation degradome of Mysticeti, we found that those proteases linked to the dentition process and enamel maintenance, \textit{KLK4} and \textit{MMP20}, were apparently functional.
Hence, as expected, we could validate that the phenotypic differences that most clearly divide Cetacea, the presence or absence of teeth, is likely to have a distinct genetic base in the presence or absence of these proteases.

These results suggest that protease gene losses have been important in the evolution of the digestive system of cetaceans.
At least in some cases, the genetic causes for these losses have been independent even between Odontoceti.
This cases of convergent evolution suggest that those events were highly favoured at the trophic level where cetaceans thrive.

Hence, the manual annotation of the degradome of sperm whales yields some insight into some of the biological peculiarities of this fascinating organism.
We have set forth hypotheses on the genetic basis of the mechanics underlining the immune response of \textit{P. macrocephalus}, reinforcing the key role of this system in the evolution of Metazoa.
In turn, this may offer information on the impact of the aquatic environment in mammalian immunological systems, and the impact of inflammation on the organism.
We have also reported several events that seem to be pivotal in the evolutionary history of this order, as suggested by convergent evolution acting on genes related to blood or skin homoeostasis.
Finally, we noted some hypothesis that would be related to Peto's paradox in these enormous animals.

Lonesome George was the iconic last member of \textit{C. abingdonii}.
Like other giant tortoises, George lived a long life.
Therefore, its genome is expected to hold clues to a different and independent solution to the problems associated with ageing.
With this hypothesis, we undertook the sequencing and annotation of this genome.

From the automatic annotation of \texttt{CheloAbing 1.0}, we found twelve expanded gene families, eight of which belong to the \quotes{extracellular exosome} GO category. This suggests that \textit{C. abingdonii} this family has been subjected to selective pressure during the evolutionary history of this species.
The correct activity of this pathway, directly impacts intercellular communication, one of the Hallmarks of ageing, in which exosomes play a crucial role.
As such, exosomes take part in many biological processes and signalling pathways related to immunity, cancer, and ageing \cite{Baixauli2014,Becker2016,Prattichizzo2017}.
In addition, this expansion be related to gigantism (and its associated cancer protection, as predicted by Peto's paradox) in giant tortoises.

Consistent with this, an additional analysis suggested that several genes involved in ERAD, \textit{BAG2}(\textit{NEF}) and \textit{UBE2J1} (\textit{Ubc6/7}) may have been subjected to positive selection.
Not only this strengthened the results from previous analysis by highlighting the apparent importance of these related pathways in giant tortoises, but one of the expanded genes, \textit{VCP}, is also a central part of this route.
In this sense, ERAD is important in the unfolded-protein response, and consequently, in the correct proteostasis of the cell, whose deregulation constitutes a hallmark of ageing \cite{Lopez-Otin2013,Scheper2015}.

Interestingly, two genes showing putative positive selection (\textit{TUBE1} and \textit{TBG1}) are related to tubulin assembly during cell cycle progression \cite{Chinen2015}.
This suggests that certain alterations affecting these genes might be related to the increase in the number of cellular divisions associated to gigantism in tortoises.
Taken together, these results suggest altered inner- and intercellular communication strategies underlying the biology of giant tortoises.

Amongst the targets of selection we found two genes previously linked to successful ageing in humans.
Specifically, expression levels of \textit{AHSG} and \textit{FGF19} have been linked to successful ageing in humans \cite{Sanchis-Gomar2015}. 
The proteins encoded by these genes are involved in glucose and lipids metabolism, meaning that their alteration could impact the regulation of nutrient sensing, one of the secondary hallmarks of ageing.
Finally, this analysis also singled out three genes, \textit{MVK}, \textit{IRAK1BP1} and \textit{IL1R2}, all of them with important roles in the modulation of the immune system.
In this regard, it is important to notice that, in addition to its role in proteostasis, ERAD is a target of viral infection, as multiple viruses depend on this process for successful delivery to the cytoplasm \cite{Morito2015}.

Taken together, this hypothesis-free analysis highlights proteostasis, metabolism regulation and immune response as key processes during the evolution of giant tortoises, and provide starting points for future work on this subject.

On the other hand, manual annotation of Lonesome George's degradome uncovered several point mutations, truncations, and copy-number-variations\nomenclature{CNV}{Copy number variations}, specific of George, Giant tortoises or testudines, that may offer interesting information.
From these results, it seems that most reptiles and specifically giant tortoises have drifted towards a situation in which the innate immune response outweighs the adaptive one.
Although several immune mediators can have dual functions both in innate and adaptive immune responses, it is thought that the innate branch of the immune system in vertebrates evolved earlier than the adaptive route \cite{Zimmerman2010}.
All multicellular organisms have some form of innate immune response, which acts as an initial step in the defence against pathogens.
Among vertebrates, Reptilia are the only ectothermic amniotes, and therefore the study of their immune system could provide new important insights into its evolution under different circumstances.

On this topic, the truncation of \textit{MEP1A}, a protease responsible for the maturation of B-lymphocytes, is expected to impact the immune response based on specific antibodies.
In addition, we have found CNVs affecting granzyme serine proteases% (table \ref{t_results_george_degradome_granzymes})
, a set of enzymes linked to the citotoxicity mediated by Natural Killers\nomenclature{NK}{Natural Killers}, one of the cellular types tasked with the innate immunological response \cite{Voskoboinik2015}.
These alterations are exclusive of the Gal\'{a}pagos giant tortoises in the case of \textit{MEP1A}, and shared only with \textit{A. gigantea} in the case of the granzyme expansion, probably playing an important role in its adaptations to size or longevity.
Other analysed species showed an expansion in the granzyme family but not so extensive as the one present in giant tortoises% (Figure \ref{app_f_results_george_degradome_granzymes})
.

Taking all of this into account, an unbalance between the two types of defence is apparent.
Of course, we should consider that this does not mean a total abandonment of the adaptive system.

Regarding proteolytical systems regulating blood homoeostasis, several proteases appear to be truncated, including \textit{F7}, \textit{F10}, and \textit{F11}% (Figure \ref{f_results_george_degradome_alignment_blood})
.
Truncations affecting these proteases usually cause Factors \textsl{VII}, \textsl{X} and \textsl{XI} deficiencies in humans, although, which points to a Dobzhansky anomaly in Gal\'{a}pagos tortoises.
Interestingly, due to several point mutations in catalytic sites (even if not specific to giant tortoises%; Figure \ref{f_results_george_degradome_alignment_blood}
), the function of PLG seems possibly lost as well.
Since the main function of its product is to aid in the dissolution of blood clots \cite{Wu2019}, its truncation adds weight to our hypothesis that blood homoeostasis systems have been under selective pressure in turtles.
Finally, \textit{PROC}, which inhibits the generation of plasmin, is apparently absent.
The absence of this gene may contribute to severe the consequences of the other alterations in the coagulation system in turtles, since its deficiency is often link to thrombosis diseases in humans \cite{Cheng2016}. 

On the subject of metabolism and diet, giant tortoises seem to have (expectedly) lost the proteases associated with dentition (\textit{KLK4} and \textit{MMP20}) as the rest of the clade, besides this, we report some interesting alterations linked to the hallmark of deregulated nutrient sensing.
Specifically, alterations relate with glucose tolerance and intake.

First, we found that neurolysin (\textit{NLN}) presents a premature stop codon, quite early in the sequence, that would abolish the function of the protein% (Figure \ref{f_results_george_degradome_alignmet_nln})
.
Interestingly, while Aldabra giant tortoise does not present a similar truncation, Mohave's dessert tortoise (\textit{G. agassizii}) does, suggesting a case of convergent evolution.
A second alteration, a duplication in pancreatic serine protease \textit{CTRB1} %(Figure \ref{f_results_george_degradome_alignment_ctrb1}
is expected to impact the process of digestion.

The development of the nervous system is a complex and intricate process in which a lot of different genes and pathways take part.
Because of this complexity, it is a metabolically expensive system to invest in, from an evolutionary point of view.
For this reason, nervous system development and derivative capabilities, are part of a trade-off.
In this regard, George's exclusive truncation of \textit{PRSS12}, giant tortoise truncations of \textit{XPNPEP1} and \textit{CNDP1}, and point variant affecting \textit{PSEN1}% (Figure \ref{f_results_george_degradome_alignment_development})
, could be related to an underdevelopment of the neural and cognitives functions when compared with other related animals \cite{Bellia2014,Jiang2015,Mitsui2013,Yoon2012}.

% HE INCLUIDO ESTOS PARRAFOS FINALES AQUI, QUE HE JUNTADO EN UNO...
In summary, the manual annotation of the degradome of sperm whales has yielded some insight on the biological peculiarities of this fascinating organism.
We have set forth hypotheses on the genetic basis of the mechanics underlining the immune response of \textit{P. macrocephalus}, reinforcing the key role of this system in the evolution of Metazoa.
In turn, this may offer information on the impact of the aquatic environment in mammalian immunological systems, and the impact of inflammation on the organism.
We have also reported several events that seem to be pivotal in the evolutionary history of this order, as suggested by convergent evolution acting on genes related to blood or skin homoeostasis.
Finally, we noted some results that might be related to the Peto paradox in these enormous animals.
In addition, the automatic and manual annotation of the genome of Lonesome George has provided information that may help in understanding the genomic basis of particular features of giant tortoises.
Considering alterations to immune-related genes, these species seem to show a different balance between innate and adaptive responses compared to mammals.
Of course, this does not mean a total abandonment of the adaptive system over the innate, but rather a more important role of the innate response.
Altogether, this analysis highlights proteostasis, metabolism regulation, cell division, and immune response as key and potentially age-related processes during the evolution of giant tortoises, and provide starting points for future work on these subjects.
% _________________________________________________________________

Therefore, the present Thesis investigates the results of the manual and automatic annotation of \textit{P. macrocephalus} and \textit{C. abingdoni} degradomes.
By searching for links between these degradomes and the hallmarks of ageing, we point out multiple genes and pathways affected by natural selection, such as \emph{loss of proteostasis}, \emph{altered intercellular communication}, and \emph{deregulated nutrient sensing}.
Hopefully, by unravelling the evolutionary history of these hallmarks in long-lived metazoans, we inch closer to greater understanding of the ageing process, and to make P. Medawar's reference to ageing as \quotes{\emph{An unsolved problem in biology}} obsolete.
