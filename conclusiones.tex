\begin{enumerate}
    \item El degradoma completo del cachalote, \textit{Physeter macrocephalus}, incluye 546 genes de proteasas, 58 de ellos clasificados como pseudogenes.
    \item Encontramos diversos eventos de pseudogenizaci\'{o}n y truncantes que afectan a proteasas, incluyendo \textit{MMP12}, \textit{TPSAB1}, y \textit{MASP2}, con funciones conocidas en el sistema inmune.
    \item Reportamos variantes truncantes en genes vinculados con el desarrollo del cáncer en \emph{P. macrocephalus}, como \textit{CASP3} y \textit{MMP7}, que se espera que eliminen el producto g\'{e}nico.
    \item Hemos ensamblado el genoma de Solitario George, el \'{u}ltimo miembro de la especie \textit{Chelonoidis abingdonii}, una de las especies de tortugas gigantes de las Islas Gal\'{a}pagos. La anotación automática de este ensamblado predijo la existencia de 27.208 genes.
    \item El degradoma completo de Solitario George se compone de 515 genes, incluyendo 63 que han sido clasificados como pseudogenes.
    \item Mediante una anotación manual y dirigida por hipótesis, del genoma de George, hemos encontrado múltiples variantes que predecimos que afecten a la actividad de proteasas con funciones conocidas en la \emph{alteración de las comunicaciones intercelulares}, una de las \quotes{marcas distintivas} integrativas del envejecimiento.
\end{enumerate}
