%\begin{flushright}
\begin{itshape}
\small

    Esta tesis es la culminación de varios años de trabajo, pero, más aún, representa la síntesis de una etapa de mi vida, y como tal, además de recoger en ella los resultados y conclusiones de los proyectos en que he participado, es necesario dedicar también unas frases a todas las personas que, de una u otra forma, han contribuido a que este barco llegue a buen puerto.

    Por ello, y en primer lugar, quiero agradecer a Carlos la oportunidad que me dio de empezar todo esto una tarde extremadamente cálida, hace ya muchos años. Gracias, Carlos, por apostar por mí en una situación en que muchos otros no lo habrían hecho, por abrirme las puertas al que, desde siempre, había sido mi sueño.
    
    Gracias a Víctor, por introducirme en el mundo de la bioinformática, por su dirección y tutela cuando necesité guía, y por dejarme espacio para crecer independientemente cuando la situación lo permitía. Gracias por tu paciencia, Víctor, sé que no siempre fue fácil.
    
    Gracias a Gloria, a Josemari, y a Xose, porque sin ellos el laboratorio no sería lo que es, por cuidar de todos nosotros en las buenas y en las malas, por su preocupación y dedicación constante. Gracias por vuestra implicación y dedicación. Gracias también a Yaiza, que con su trabajo hace el de los demás más fácil. Gracias por intentar animarnos y ayudarnos siempre.
    
    Gracias a todos los compañeros de laboratorio, con los que tantas horas he compartido. Gracias por toda la ayuda prestada y por tantos buenos ratos, por los cafés, las conversaciones, y por tantas tardes de sesiones músicales.
    
    Thanks to Dr. Kevin Howe for letting me visit his lab for three wonderful months. Thank you for your help and willingness. Thanks also to Fergal and his team (especially to Leanne), for your patience, your advice and guidance, and for making me feel welcomed and \emph{cozy}. You helped me to learn a different way of doing science and for that I'm (again) grateful.
    
    Y por supuesto, gracias a la Universidad de Oviedo, al Instituto Universitario Oncológico del Principado de Asturias, y al ministerio de Salud por su continuado apoyo financiero y sus ayudas durate la realización de esta tesis.

    \bigskip
    
    Pero no todo iba a ser trabajar, así que tengo que dedicar unas palabras de agradecimiento especial a un grupo que pese a haber nacido entre las paredes del labo, ha trascendido con creces a lo largo de los años. Gracias a las personas que pasaron de no atreverse a compartir coche en un viaje de 200 kilómetros a venirse conmigo al otro lado del globo. No son palabras vacías si os digo que sin vosotros esto no habría sido posible. No solo por el constante apoyo y los buenos momentos, sino por compartir los malos (y los muy malos). Por todas las fiestas, las \quotes{autoinvitaciones}, los viajes y las escapadas, mil gracias.
    
    Gracias también a mi \quotes{cohorte de Sandramandra}, gracias por haber estado ahí desde tiempos inmemoriales. Por las innumerables horas de aventuras imaginarias en tierras desconocidas, por todas las visitas (casi improvisadas) a mi país adoptivo, las reuniones (algo más planificadas) en el país bávaro, y los (de momento inexistentes) viajes a las tierras de la lengua incomprensible. Y, por supuesto, por las \quotes{tertulias científicas} durante las noches de vermú. Gracias por todo.
    
    A mi \quotes{equipo de IT}, gracias. Gracias por que con vuestros chistes y provocaciones me disteis la determinación que necesitaba para avanzar en un campo que me era desconocido, y con vuestros consejos y eterna disposición, me ayudasteis a progresar en el mismo. Gracias por estar \quotes{aquí}, por las noches de peli, por las sesiones de turismo de \quotes{aperitivo}, por los campings y por todos los buenos momentos.

    No puedo terminar sin darle las gracias a mi familia, por su constante e incondicional apoyo. A mis padres, sin quienes no habría llegado tan lejos. Gracias por llevarme a la universidad, por ayudarme a avanzar curso a curso, y por seguir a pie del cañón día a día. También a mi hermana, compañera en muchos viajes (no solo de carretera), a mis tíos y tías, por vuestro interés en mi trabajo y vuestros ánimos, a mis primos por todos los buenos ratos de esparcimiento, y, de una forma muy especial, gracias a mis abuelas, por todo vuestro apoyo y amor.

    \bigskip

    Finalmente, me gustaría dedicarle la presente tesis a mis abuelos Paco y Pepe, que de una forma u otra me han inspirado a tomar este camino y han hecho que me sienta orgulloso de haberlo seguido. Gracias por ayudarme a estar aquí hoy.

\end{itshape}
%\end{flushright}
