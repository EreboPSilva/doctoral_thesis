\begin{sidewaysfigure}[ht!]                                                                    
\begin{center}                                                                             
\includegraphics[width=\textwidth]{figures/George_summary.pdf}                          
    \caption[Testudines truncations and alterations in proteases]{\footnotesize Testudines truncations and alterations in proteases. Each mark represents an adaptative event, either truncations or duplications (represented as a doble mark), in the branch in which data suggests it occurred. Events of convergent evolution are represented by a doted line joining the pertinent marks. Those marks filled with \textcolor{myora1}{\textbf{orange}} are those that, although mentioned in the main text, most likely played a late role once a main event of adaptation had already occured (said event will be represented by \textcolor{myaqu1}{\textbf{green}} as the rest). \hsap; \texttt{Saur*} represents other Sauropsids (such as \textit{G. gallus}, or \textit{A. carolinensis}); \psin; \gaga; \agig; \texttt{Cont*} reffers to the continental outgroups of the Gal\'{a}pagos giant tortoises, \textit{C. denticulata} y \textit{C. carbonaria}; \texttt{C.sp*} (\textit{C. sp}) refers to all Gal\'{a}pagos giant tortoises but \textit{C. abingdoni}, which is referenced as \texttt{cabi}. The tree itself is based in the relations shown by \quotes{Time tree} \cite{Kumar2017}.}
\label{f_results_george_summary}                                                                
\end{center}                                                                               
\end{sidewaysfigure}
